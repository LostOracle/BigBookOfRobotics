% !TEX root = HowToRobot.tex

\chapter{Demo Day}
\label{chap:demoday}

So, you've got five minutes. Here comes Dr. Riley with a tour group and you weren't ready. Never fear! These instructions will have you up and running with an RC robot in half that time.

You'll need the following:

\begin{itemize}
\item{USB-to-Serial adapter (optional)}
\item{USB-USBmini cable}
\item{PS3 Controller}
\item{Computer with Wifi running Fedora 21 with ROS and joy installed}
\end{itemize}

Start the timer. If you set up a static ip address for the odroid as shown in Chapter \ref{chap:odroidsetup}, skip step 2.
\begin{enumerate}
\item{Power on robot}
\item{Plug the PS3 controller in to the computer}
\item{Plug/Unplug the PS3 controller and push the button until the player 1 light stays on}
\item{Connect wirelessly to the robot}
\item{Serial in to the Odroid to find the ip address - optional}
\item{Source your ROS install}
\begin{lstlisting}[language=bash]
  $ source /opt/ros/indigo/setup.bash
\end{lstlisting}
\item{Find your own ip address - it'll be a 192.168 address}
\item{Export IPs}
\begin{lstlisting}[language=bash]
  $ export ROS_IP=<your ip>
  $ export ROS_MASTER_URI=http://<robot ip>:11311
\end{lstlisting}
\item{Run the joy node locally}
\begin{lstlisting}[language=bash]
  $ rosrun joy joy_node joy:=/<robot>/joy
\end{lstlisting}
\end{enumerate}

In the last step <robot> is the name of the robot used in the the launchfile as a namespace. See Chapter \ref{chap:rosarchitecture} for details.

Done. Probably. You should be able to move the joysticks on the PS3 controller and get the robot to move at this point. If that doesn't work, you probably missed a step during setup. Or I missed a step with this guide. If that's the case, let me know so I can update it.