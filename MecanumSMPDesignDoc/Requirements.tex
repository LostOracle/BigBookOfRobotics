% !TEX root = DesignDoc.tex

\chapter{Requirements}
\label{chap:requirements}

As demonstration and research robots, the SMP and Mecanum robots don't have any specific external requirements. However, there are several more subjective internal requirements for this iteration of the robots.\\

The Mecanum and SMP robots shall:

\begin{itemize}
  \item{Provide a Stable Research Platform}
  \item{Be Good Examples of ROS Design}
  \item{Separate Electrically Noisy Components}
  \item{Be Easily Re-configurable}
  \item{Be Easily Maintainable and Accessible}
\end{itemize}

\section{Provide a Stable Research Platform}

This requirement states that this iteration of the robots shall be useful for demos, research, and general learning by the members of the robotics team or MCS department senior design teams. One important aspect of fulfilling this requirement is supplying a versatile platform. This means the robots shall be designed in such a way that various sensors can be mounted in multiple useful positions.
The typical sensor payload for the SMP and Mecanum robots includes:

\begin{itemize}
  \item{ASUS RGBD Vision Sensor}
  \item{LiDAR}
  \item{IMU}
\end{itemize}

In keeping with the spirit of versatility the robots shall be over-engineered such that additional sensors can be added without requiring electrical redesign.

\section{Be Good Examples of ROS Design}

This requirement states that the Mecanum and SMP robots are intended to be good examples of ROS design. Great care shall be taken to use as many ROS Good Practices as possible.\\

These Good Practices shall include but not be limited to:

\begin{itemize}
	\item{Utilitize the Systemd Services utility for auto-boot}
	\item{Utilize the udev utility for mounting devices}
	\item{Use dedicated driver nodes}
	\item{Use no custom message types}
	\item{Write highly generic nodes where possible}
\end{itemize}

\section{Separate Noisy Components}

This requirement states that electrically noisy components shall be isolated from the rest of the system, particularly the Odroid and sensors. One of the problems these robots experienced in the past was a condition in which the motor controllers would unmount from the Odroid. The cause was ruled to be noise generated by either the motor controllers themselves or the motors creating voltage spikes that scared the Odroid hardware into unmounting the USB devices to protect itself. To mitigate this problem, two measures shall be taken. First, motor controllers will be selected which can be connected over serial lines. The big advantage in doing so is that putting an FTDI between the motor controller and the Odroid USB port adds a layer of electrical isolation.

\section{Easily Re-configurable}

This requirement states that the robots shall be designed in such a way that a minor reconfiguration does not require redesigning the entire robot. In particular, the robots shall be over-engineered with respect to torque, power requirements, and electrical connections such that additional sensors can later be added without requiring electrical redesign.

\section{Easily Maintainable and Accessible}

This requirement states that the robots shall be designed in such a way that routine maintenance, such as battery recharging, can be done easily. In particular, the robot shall be designed such that it can be opened for maintenance without requiring the removal of the sensor payload, computation devices, or wireless capability.