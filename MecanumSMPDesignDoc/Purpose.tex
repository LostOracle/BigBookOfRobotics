% !TEX root = DesignDoc.tex

\chapter{Purpose}
\label{chap:purpose}

It should be noted above all that the SMP type robots are capable of indoor and light outdoor use, while the Mecanum is unsuitable for outdoor use. Beyond that, the SMP and Mecanum robots are intended for three primary uses:

\begin{itemize}
\item{Learning and Research}
\item{Functional Examples}
\item{Demos}
\end{itemize}

\section{Learning and Research}

One primary problem with building a student-led team based around a complex and competitive field like robotics is the long ramp-up time for new members and the perpetual attrition of more experienced members. To put it another way, by the time a new member learns enough to start pushing the limits on what the team can do, it is already time for them to graduate. In addition, the loss of conceptual understanding is devastating in a field where we generally are more focused on algorithms and integration than producing a physical deliverable. Add to that how quickly the field changes, and the cyclic nature of academia becomes crippling. The SMP and Mecanum robots are to provide three good examples of a functional, yet simple implementation of basic robotics principles including mechanical design (sort of), two types of kinematics, electrical design, and - principally - ROS architecture.

The hope is that having these examples readily available will allow new members or senior design teams working toward a robotics-related project to get up and running quickly. So often, a new group will get bogged down in the details of building and wiring their robot, that they run out of time before they get a chance to actually deal with the more "interesting" problem that they had actually intended to tackle from the beginning. Having working robots available allows the teams and/or new members to skip right to their intended topic of interest, whether it be modifications of the mechanical or electrical design, control systems, vision, navigation, or autonomy.

\section{Functional Examples}

ROS itself isn't fundamentally all that complex, but robots are. Another intent when working on these robots was to provide a "minimal working set" of ROS packages that can be used as a spring-board for future ROS development. It doesn't take \textit{that} many pieces to make a robot in ROS, but knowing what exactly those pieces are is a daunting task when starting from nothing. By using the very simplistic design of the SMP and Mecanum robots as an example, the task of dealing with a robot in ROS becomes much more approachable.

This is doubly true with mechanical and electrical design. A lot of parts have to come together to make a robot, and a novice to robotics design will inevitably feel overwhelmed when attempting to design a robot from nothing.

\section{Demos}

One of the Robotics team's main responsibilities as good citizens of the MCS department is making a good showing at presentations or demos for the department. In the past, this has usually been the domain of between one and three robotics team members who knew enough about the robots to cobble something together at the last minute before the demo. With three readily available robots on hand, these demo days should be more interesting, more reliable, and less stressful for everyone involved. In addition, having multiple demo robots means that the presenters have backups in case one fails, and can attempt more interesting demonstrations knowing that they have a fall-back plan.

\section{Not Competition}

The intent of the SMP and Mecanum robots is not that they will be used in robotics competitions. They should always be kept in working condition, and not overhauled unless for the express purpose of providing a more robust application of the three goals mentioned above. The team should go through the entire design process - even if that means buying a new chassis - rather than skipping it in favor of a pre-existing robot. Even though it means more work and a larger risk of failure, the design process - and the occasional failure - are invaluable learning experiences.
