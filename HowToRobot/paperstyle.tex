
\usepackage[width=6.5in, height=9.0in, top=1.0in, papersize={8.5in,11in}]{geometry}
\usepackage[pdftex]{graphicx}
\DeclareGraphicsExtensions{.pdf,.png,.jpg}
%\usepackage{draftwatermark}
\usepackage{amsmath}
\usepackage{amsthm}
\usepackage{amssymb}
%\usepackage{txfonts}
\usepackage{textcomp}
%\usepackage{amsthm}
%\usepackage{array}
%\usepackage{datetime}
%\usepackage{anyfontsize}
\usepackage{cite}
\usepackage{enumitem}
\usepackage{t1enc}
\usepackage[section,subsection]{extraplaceins}   %%%  \FloatBarrier
\usepackage[all]{xy}
\usepackage{fancyhdr}
\usepackage{hyperref}
\usepackage{verbatim}
\usepackage{algorithm}
\usepackage{algorithmic}
\usepackage{makeidx}
\usepackage{multicol}
\usepackage{multirow}
\usepackage{color}
\usepackage{rotating}
\usepackage{wrapfig}
\usepackage{tikz}
\usetikzlibrary{shapes.geometric, arrows}
%\usepackage{tabularx}
\usepackage{xcolor}
%\usepackage{framed}
\usepackage{xspace}
\usepackage{listings}
\usepackage{subcaption}
\lstset{language=python,frame=ltrb,framesep=5pt,basicstyle=\normalsize,
 keywordstyle=\ttfamily\color{DarkRed},
%morecomment=[n][\textbf]{In\ [}{]\:},
%morecomment=[n][\textbf]{Out\ [}{]\:},
morecomment=[s][\color{blue}]{In\ [}{]\:},
morecomment=[s][\color{red}]{Out[}{]\:},
identifierstyle=\ttfamily\color{DarkBlue}\bfseries,
commentstyle=\color{OliveGreen},
stringstyle=\ttfamily,
showstringspaces=false,tabsize = 3}

\lstdefinelanguage{shell} {
commentstyle = \color{black},
keywordstyle = \color{black},
stringstyle = \color{black},
identifierstyle = \color{black},
morecomment=[s][\color{blue}]{In\ [}{]\:},
morecomment=[s][\color{red}]{Out[}{]\:},
 }

\newtheorem{thrm}{Theorem}
\newtheorem{lem}[thrm]{Lemma}
\newtheorem{cor}[thrm]{Corollary}
\newtheorem{rem}[thrm]{Remark}
\newtheorem{defn}[thrm]{Definition}
\newtheorem{exmpl}[thrm]{Example}

% this gives a little box for the end of a proof:
%
\def\endthrmbox{$\sqsubset \!\!\!\! \sqsupset$}

\newcommand{\dis}{\displaystyle}
 \def      \RR             {{\mathbb R}} 
        \def      \NN             {{\Bbb N}} 
        \def      \QQ             {{\Bbb Q}} 
        \def      \CC             {{\Bbb C}} 
        \def      \ZZ             {{\Bbb Z}} 
 
 
        \def       \a              {{\alpha}} 
        \def       \b              {{\beta}} 
        \def       \d              {{\delta}} 
        \def       \D              {{\Delta}} 
        \def         \e              {{\varepsilon}} 
        \def         \g              {{\gamma}} 
        \def         \G              {{\Gamma}} 
        \def       \l              {{\lambda}} 
        \def       \L              {{\Lambda}} 
        \def        \m               {{\mu}} 
        \def         \n              {{\nabla}} 
        \def       \var          {{\varphi}} 
        \def         \s              {{\sigma}} 
        \def       \Sig          {{\Sigma}} 
        \def       \Om          {{\Omega}} 
 
        \def       \t              {{\tau}} 
        \def         \th             {{\theta}} 
        \def       \O              {{\Omega}} 
        \def       \o              {{\omega}} 
        \def         \z              {{\zeta}} 
       \def        \P             {{\Phi}} 
       \def        \p             {{\phi}} 
        %Other macros 
 
        \def       \iy              {{\infty}} 
        \def         \pa             {{\partial}} 
        \def         \div           {{\rm div}} 
         \def       \na            {{\nabla}} 
 



\newcommand{\pythonlogo}{
\\[-2mm] \begin{picture}(0,0)
\put(-40,-40){\includegraphics[scale=0.25]{./Figures/pythonlogo.png}}
\end{picture}
}

\newcommand{\clogo}{
\\[-2mm] \begin{picture}(0,0)
\put(-30,-30){\includegraphics[scale=0.2]{./Figures/clogo.png}}
\end{picture}
}

\newcommand{\roslogo}{
\\[-2mm] 
\begin{picture}(0,0)
\put(-30,-30){\includegraphics[scale=0.2]{./Figures/roslogo.png}}
\end{picture}
}


\tikzstyle{master} = [rectangle, draw, text width=6em, text centered, minimum
height=3em]
\tikzstyle{node} = [rectangle, draw, text width=6em, text centered, rounded
corners, minimum height=3em]

\newtheorem{summary}{Summary:}
\newtheorem{example}{Example:}[section]

\definecolor{OliveGreen}{cmyk}{0.64,0,0.95,0.40}
\definecolor{DarkBlue}{cmyk}{0.76,0.76,0,0.20}
\definecolor{DarkRed}{cmyk}{0,1,1,0.45}


\def      \RR             {{\mathbb R}} 
\def      \DS            {\displaystyle} 

\setlength{\oddsidemargin}{0mm} 
\setlength{\evensidemargin}{0mm} 

%\SetWatermarkLightness{0.975}
%\SetWatermarkScale{6}
%\SetWatermarkText{\includegraphics{test.png}}

\pagestyle{fancy}
\renewcommand{\chaptermark}[1]{\markboth{#1}{}}
\renewcommand{\sectionmark}[1]{\markright{\thesection\ #1}}
\fancyhf{}
\fancyhead[LE,RO]{\bfseries\thepage}
\fancyhead[LO]{\bfseries\rightmark}
\fancyhead[RE]{\bfseries\leftmark}
\renewcommand{\headrulewidth}{0.5pt}
\renewcommand{\footrulewidth}{0pt}
\addtolength{\headheight}{0.5pt}
\setlength{\footskip}{0in}
\renewcommand{\footruleskip}{0pt}
\fancypagestyle{plain}{%
\fancyhead{}
\renewcommand{\headrulewidth}{0pt}
}


\definecolor{color02}{rgb}{0.18,0.35,0.59}
\definecolor{color03}{rgb}{0.44,0.59,0.82}
\definecolor{color06}{rgb}{0.35,0.35,0.35}


\definecolor{MSBlue}{rgb}{.204,.353,.541}
\definecolor{MSLightBlue}{rgb}{.31,.506,.741}
\definecolor{MSBlue1}{rgb}{0.18,0.35,0.59}
\definecolor{MSBlue2}{rgb}{0.44,0.59,0.82}
\definecolor{MSBlue3}{rgb}{0.35,0.35,0.35}

\usepackage{titlesec}
\titleformat{\chapter}[display]
%{\normalfont\bfseries\color{MSBlue1}}    %\normalfont\bfseries\filcenter}
{\normalfont\bfseries}    %\normalfont\bfseries\filcenter}
{\LARGE\thechapter}
{1ex}
{\titlerule[2pt]
\vspace{2ex}%
\LARGE}
[\vspace{1ex}%
{\titlerule[2pt]}]



\date{\today}

